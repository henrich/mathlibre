\documentclass[a4]{article}
\usepackage{url,listings}
\newcommand{\ir}{ir053\_portable}
\newcommand{\vboxwin}{VirtualBox-5.2.8-121009-Win.exe}
\newcommand{\vboxmac}{VirtualBox-5.2.8-121009-OSX.dmg}
\newcommand{\vm}{{\em mathlibre2018-en-vm}}
\lstset{
  basicstyle=\small\ttfamily,
  keywordstyle=\color{black}\bfseries\underbar,
  stringstyle=\ttfamily,
  showstringspaces=false,
  frameround=tttt,
  frame=trbl
}
\title{An introduction for making the virtual environment of MathLibre}
\author{Tatsuyoshi HAMADA}
\begin{document}
\maketitle
MathLibre is a live Linux, it's bootable from DVD.
But, you may think it's inconvenient for the slow speed of DVD.
When you use virtual machine on your host operating system(Windows or Mac OS) and
putting the ISO image file on your hard disk,
you can boot MathLibre computer environment with your operating system
at the same time.
You can use persistent home directory and install your favorite Debian
packages with this environment.

We use an open source virtual machine,
VirtualBox\footnote{\url{http://www.virtualbox.org/}}.
VirtualBox is an application on your operating system,
you can use your favorite operating systems on it.
Now, it's developed by Oracle Corporation.
We prepare setting files for making virtual environment 
in {\em vbox} folder of MathLibre DVD.

\section{Windows case}
\subsection{Installing VirtualBox}
We introduce how to boot MathLibre on Windows.
\begin{enumerate}
 \item Download the current version of VirtualBox for Windows from the
       website and install with administration mode.
 \item Please unzip the file \vm.zip{} to your favorite place, for example
       in your {\em Documents} folder. We set the name of extracted folder \vm{}.
\end{enumerate}

\subsection{Getting MathLibre ISO image file}
You have two choices for getting MathLibre ISO image,
``Download'' and ``Making from DVD''.

\subsubsection{Download ISO image file}
If you can use the Internet connection,
please download the current ISO image from ftp site\footnote{\url{ftp://ftp.mathlibre.org/pub/mathlibre/}}.
It has around 4GB, it takes some minutes or hours.
When you finish downloading the ISO file,
move it in the folder \vm{} and rename this ISO file to
{\em mathlibre.iso}. 

\subsubsection{Making ISO image file from DVD}
When you have MathLibre DVD, 
you do not need the Internet connection for getting ISO image.
MathLibre DVD is including the Windows application, InfraRecorder\footnote{\url{http://infrarecorder.org/}}.
We can make ISO image file with MathLibre DVD.
\begin{enumerate}
 \item Unzip the file \ir.zip{} in MathLibre DVD.
 \item Execute infrarecorder.exe in the folder \ir.
 \item Select Read Disc.
 \item You can find DVD drive as Source:.
 \item Select ISO image file {\em mathlibre.iso} in the folder \vm{} for Image file:.
 \item Clicking OK button, it will start to copy DVD in your hard disk.
 \item It takes about 15 minutes for making 4GB ISO image.
\end{enumerate}

\subsection{Booting virtual machine}
When you finish every procedures,
execute VirtualBox with double clicking \vm.{\em vbox} in \vm{} folder.
Click the Start icon and push Enter key, you can enjoy mathematical software.

\section{Mac OS X case}
\subsection{How to install VirtualBox}
We introduce how to boot MathLibre on Mac OS.
You need to use the Internet connection.
\begin{enumerate}
 \item Download the current version of VirtualBox for Mac OS from the
       website and install with administration mode.
 \item Download \vm.zip{} from
       \url{ftp://ftp.mathlibre.org/pub/mathlibre/} and extract it to
       your favorite place, for example home directory, we set this
       directory's name \vm.
\end{enumerate}

\subsection{Getting MathLibre DVD ISO image}
Download the current ISO image from
\url{ftp://ftp.mathlibre.org/pub/mathlibre/}
and move it to the folder \vm{}. You need to rename ISO file to {\em mathlibre.iso}.

\subsection{Booting virtual machine}
\begin{enumerate}
 \item When you finish your every procedures, execute VirtualBox.
 \item Select \vm{} in \vm.{\em vbox} with the menu of ``Virtual Machine'' $>$ ``Add''.
 \item Click the Start icon and push Enter key, you can enjoy mathematical software.
\end{enumerate}

\section{How to make Shared Folders}
\begin{enumerate}
 \item Click the Settings icon of VirtualBox,
       and select Shared Folders, you can find
       ``\verb|+| folder'' icon, click this icon and set Folder Path and
       Folder Name, for example, the folder name is \verb|X|,
       you should keep in mind, check Auto-mount box and push OK button.
 \item  In virtual machine, this shared folder is mounted with
	\verb|/media/sf_X|. We will create a symbolic link to this folder with
	the name \verb|Y|. It is similar to shortcut in Windows. You can make
	it with the following command:
\begin{lstlisting}
ln -s /media/sf_X Y
\end{lstlisting}
\item Logout and login with username:\underline{user} and password:\underline{live}. You can access to the shared folder with this command in a terminal. 
\begin{lstlisting}
 ls Y
\end{lstlisting}
Of course, you are able to use a file manager PCManFM, so you can share
files with this shared folder between host operating system and
      MathLibre virtual machine.
\end{enumerate}

\section{Some tips}
If you have some troubles for booting MathLibre with VirtualBox,
please check the BIOS menu of ``Intel Virtualization Technology'' is
valid or not.
\end{document}
