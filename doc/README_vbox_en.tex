\documentclass[a4]{article}
\usepackage{url,listings}
\newcommand{\ir}{ir053\_portable\_x64}
\newcommand{\vboxwin}{VirtualBox-4.3.6-91406-Win.exe}
\newcommand{\vboxmac}{VirtualBox-4.3.6-91406-OSX.dmg}
\newcommand{\vm}{{\em mathlibre2014-vm}}
\lstset{
  basicstyle=\small\ttfamily,
  keywordstyle=\color{black}\bfseries\underbar,
  stringstyle=\ttfamily,
  showstringspaces=false,
  frameround=tttt,
  frame=trbl
}
\title{An introduction for making the virtual environment of MathLibre}
\author{Tatsuyoshi HAMADA}
\begin{document}
\maketitle
MathLibre is a live linux, it's bootable from DVD.
But, you may think it's inconvenient for the slow speed of DVD.
When you use virtual machine on your operating system(Windows or Mac OS) and
putting the ISO image file on your harddisk,
you can boot MathLibre computer environment with your operating system
at the same time.
You can use parsistent home directory and install your favorite Debian
packages with this environment.

We use an open source virtual machine,
VirtualBox\footnote{\url{http://www.virtualbox.org/}}.
VirtualBox is an application on your operating system,
you can use your favorite operating systems on it.
Now, it's developed by Oracle company.
We prepare setting files for making virtual environment 
in {\em vbox} folder of MathLibre DVD.

\section{Windows case}
\subsection{Installing VirtualBox}
We introduce how to boot MathLibre on Windows.
\begin{enumerate}
 \item Please check the file \vboxwin{} in MathLibre DVD, or download the
       current version of VirtualBox and install with administration mode.
 \item Please unzip the file \vm.zip{} to your favorite place, for example
       in your {\em Documents}. We set the name of extracted folder {\em
       mathlibre2014-vm}. 
\end{enumerate}

\subsection{Getting MathLibre ISO image file}
You have two choices for getting MathLibre ISO image,
``Download'' and ``Making from DVD''.

\subsubsection{Download}
If you can use the Internet connection,
please download the current ISO image from ftp site\footnote{\url{ftp://ftp.mathlibre.org/pub/mathlibre/}}.
It has around 4GB, it takes some minutes or hours.
When you finish downloading the ISO file,
move it in the folder \vm{} and rename this ISO file to
{\em mathlibre2014.iso}. 

\subsubsection{Making from DVD}
When you have MathLibre DVD, 
you do not need the Internet connection for getting ISO image.
MathLibre DVD is including the Windows application, InfraRecorder\footnote{\url{http://infrarecorder.org/}}.
We can make ISO image file with MathLibre DVD.
\begin{enumerate}
 \item Unzip the file \ir.zip{} in MathLibre DVD.
 \item Execute infrarecorder.exe in the folder \ir.
 \item Select Read Disc.
 \item You can find DVD drive as Source:.
 \item Select ISO image file in \vm{} for Image file:.
 \item Push OK button, starting to copy DVD in your harddisk.
 \item It takes about 15 minutes for making 4GB ISO image.
\end{enumerate}

\subsection{Booting virtual machiine}
When you finish every procedures,
execute VirtualBox with double clicking \vm.{\em box} in \vm{} folder.
Push the boot button, you can enjoy mathematical software.

\section{MacOS X case}
\subsection{How to install VirtualBox}
We introduce how to boot MathLibre on Mac OS.
You need to use the Internet connection.
\begin{enumerate}
 \item Download the current Mac OS version from
       \url{http://www.virtualbox.org/} and install it.
 \item Download \vm.zip{} from
       \url{ftp://ftp.mathlibre.org/pub/mathlibre/} and extract it to
       your favorite place, for example home directory, we set this
       directory's name \vm.
\end{enumerate}

\subsection{Getting MathLibre DVD ISO image}
Download the current ISO image from
\url{ftp://ftp.mathlibre.org/pub/mathlibre/}
and move it to the folder \vm{}. You need to rename ISO file to {\em mathlibre2014.iso}.

%\subsubsection{DVDから作成}
%ネットワーク環境がない場合でもMathLibre DVDがあれば,ISOイメージを作成で
%きます.MacOS Xのユーティリティの一つ,ディスクユーティリティを用います. 
%\begin{enumerate}
% \item 「Launchpad」の「ユーティリティ」フォルダにある「ディスクユーティリティ」を開きます.
% \item DVD ディスクを光学ドライブにセットし,左側のリストでそのディスクを選択します.
% \item 「ファイル」>「新規」>「<ディスク名>からのディスクイメージ」と選択します.
% \item ディレクトリ \vm 内に mathlibre2014.isoというファイル名で作成します.
% \item 「イメージフォーマット」ポップアップメニューからオプションを選択します. 
% \item 「保存」をクリックします.
% \item 約4GBのファイルを作成するのに約15分ほどかかります.
%\end{enumerate}

\subsection{Booting virtual machine}
\begin{enumerate}
 \item When you finish your every procedures, execute VirtualBox.
 \item Select \vm{} in \vm.{\em vbox} with the menu of ``Virtual Machine'' $>$ ``Add''.
 \item Pushing Boot button, you can enjoy mahtematical software.
\end{enumerate}

\section{Making 共有フォルダの作成}
\begin{enumerate}
 \item VirtualBoxの設定アイコンから「共有フォルダー」の\verb|<+>|アイコンをクリックし,「フォルダーのパス」で共有したいフォルダーを指定,表示される「フォルダー名」(この名前は変更可能です.仮に\verb|X|とします.)を控え,「自動マウント」(および,もしあれば「永続化する」)にチェックを入れ,「OK」をクリックしてください.
 \item 仮想マシン上では,\verb|/media/sf_X| にマウントされます(\verb|X|は先に控えたフォルダー名).この共有フォルダーを仮に\verb|Y|という名前でシンボリックリンク(ショートカットのようなもの)を作成します.端末で以下の命令を入力してください.
\begin{lstlisting}
ln -s /media/sf_X Y
\end{lstlisting}
\item ログアウト後,再ログイン(ユーザ名:user, パスワード live)すれば
\begin{lstlisting}
 ls Y
\end{lstlisting}
のようにアクセスできます.もちろんファイルマネージャPCManFMでも利用可能です.
\end{enumerate}

%\begin{thebibliography}{9}
% \bibitem{diskutil} Disk Utility 12.x: ディスク、CD、または DVD を複製する, \url{http://support.apple.com/kb/PH5846?viewlocale=ja_JP}
%\end{thebibliography}
\end{document}
