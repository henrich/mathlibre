\documentclass[a4j]{ltjsarticle}
\usepackage{url,listings,fancyhdr}
\newcommand{\ir}{ir053\_portable\_x64}
\newcommand{\vboxwin}{VirtualBox-5.0.14-105127-Win.exe}
\newcommand{\vboxmac}{VirtualBox-5.0.14-105127-OSX.dmg}
\newcommand{\vm}{mathlibre-ja-vm}
\pagestyle{fancy}
\lstset{
  basicstyle=\small\ttfamily,
  keywordstyle=\color{black}\bfseries\underbar,
  stringstyle=\ttfamily,
  showstringspaces=false,
  frameround=tttt,
  frame=trbl
}
\lhead{\Large MathLibre: 仮想環境での利用方法}
\chead{}
\rhead{MathLibre Project}
\begin{document}
%\maketitle
MathLibreはDVDから起動できるLinuxです.しかし,DVDはアクセス速度が遅く
普段使いには不便かもしれません.DVDをISOイメージファイルとしてハードディスクに置
き,仮想化ソフトウェアを用いて仮想環境を作ると,
WindowsやMacを使いながら同時に利用することができます.
以下に紹介する仮想環境 \vm.zip は既に設定済みですので,
継続的なホームディレクトリやパッケージの追加インストールも可能です.
ISOイメージファイルとホームディレクトリのため,約12GB以上の空き容量が必要です.

仮想化ソフトウェアとして,オープンソースソフトウェアの VirtualBox を利用します.
VirtualBoxは既存のオペレーティングシステム上のアプリケーションの一つとし
てインストールされ,この中で追加のオペレーティングシステムを動かすことが
できます.

\section{Windowsの場合}
仮想環境作成に必要なファイルは\underline{DVD内のフォルダvboxにまとめてあります.}\paragraph{VirtualBoxのインストール}
ここではWindows上でMathLibreを動かす方法を解説します.
\begin{enumerate}
 \item MathLibre DVD内の \vboxwin もしくはネット上から最新版を取得し,管理者権限で実行してインストールします.
 \item MathLibre DVD内のファイル \vm.zip をドキュメント等の適当な場所に展開します.展開されたフォルダを \vm とします.
\end{enumerate}

\paragraph{DVDのISOイメージを取得}

ISOイメージの取得方法として「ダウンロード」と「DVDから作成」の2通りの方法があります.
\subparagraph{「ダウンロードして取得」}

ネットワーク環境が整っている場合には,
\url{ftp://ftp.mathlibre.org/pub/mathlibre/}より
最新版のISOイメージファイル(約4GB)をダウンロードしてください.
ダウンロードしたら,先ほどのフォルダ \vm 内に移動させて,ファイル名をmathlibre.isoに変更してください.

\subparagraph{「DVDから作成」}
ネットワーク環境がない場合でもMathLibre DVDがあれば,ISOイメージを作成できます.
DVDにWindowsアプリケーションInfraRecorderを同梱しています.
InfraRecorderを用いてハードディスク上にMathLibre DVDのISOイメージを作成
します.
\begin{enumerate}
 \item MathLibre DVD内の\ir.zipを適当な場所に展開します.
 \item \ir 内にあるinfrarecorder.exeを実行します.
 \item Read Discを選択します.
 \item Source:としてDVDドライブが選択されているはずです.
 \item Image file:として,先ほどのフォルダ \vm 内にmathlibre.isoを指定します.
 \item OKボタンを押すと,DVDの複製が始まります.
 \item 約4GBのファイルを作成するのに約15分ほどかかります.
\end{enumerate}

\paragraph{仮想環境の起動}
すべての作業を終えたら,\vm 内の \vm.vbox をダブルクリックして
VirtualBox を起動してください.起動ボタンを押すと仮想環境が起動します.

\section{MacOS Xの場合}
次にMacOS X上でMathLibreを動かす方法を解説します.
最近のMacは標準では光学ドライブを搭載していませんのでネットワークを用いる方法を紹介します.
\paragraph{VirtualBoxのインストール}
\begin{enumerate}
 \item \url{http://www.virtualbox.org/}からMacOS X用最の新版を取得し,実行してインストールします.
 \item \url{ftp://ftp.mathlibre.org/pub/mathlibre/}から \vm.zip をダウンロードし,ホームディレクトリ等の適当な場所に展開します.展開されたディレクトリを \vm とします.
\end{enumerate}

\paragraph{DVDのISOイメージを取得}
\url{ftp://ftp.mathlibre.org/pub/mathlibre}より
最新版のISOイメージファイル(約4GB)をダウンロードしてください.
ダウンロードしたら,先ほどのディレクトリ \vm 内に移動させて,ファイル名をmathlibre.isoに変更してください.

%\subparagraph{DVDから作成}
%ネットワーク環境がない場合でもMathLibre DVDがあれば,ISOイメージを作成で
%きます.MacOS Xのユーティリティの一つ,ディスクユーティリティを用います. 
%\begin{enumerate}
% \item 「Launchpad」の「ユーティリティ」フォルダにある「ディスクユーティリティ」を開きます.
% \item DVD ディスクを光学ドライブにセットし,左側のリストでそのディスクを選択します.
% \item 「ファイル」>「新規」>「<ディスク名>からのディスクイメージ」と選択します.
% \item ディレクトリ \vm 内に mathlibre.isoというファイル名で作成します.
% \item 「イメージフォーマット」ポップアップメニューからオプションを選択します. 
% \item 「保存」をクリックします.
% \item 約4GBのファイルを作成するのに約15分ほどかかります.
%\end{enumerate}

\paragraph{仮想環境の起動}
\begin{enumerate}
 \item すべての作業を終えたら VirtualBox を起動してください.
 \item メニューから「仮想マシン」>「追加」で\vm 内の \vm.vbox を選択してください.
 \item 起動ボタンを押すと仮想環境が起動します.
\end{enumerate}

\section{共有フォルダの作成}
\begin{enumerate}
 \item VirtualBoxの設定アイコンから「共有フォルダー」の\verb|<+>|アイコンをクリックし,「フォルダーのパス」で共有したいフォルダーを指定,表示される「フォルダー名」(この名前は変更可能です.仮に\verb|X|とします.)を控え,「自動マウント」(および,もしあれば「永続化する」)にチェックを入れ,「OK」をクリックしてください.
 \item 仮想マシン上では,\verb|/media/sf_X| にマウントされます(\verb|X|は先に控えたフォルダー名).この共有フォルダーを仮に\verb|Y|という名前でシンボリックリンク(ショートカットのようなもの)を作成します.端末で以下の命令を入力してください.
\begin{lstlisting}
ln -s /media/sf_X Y
\end{lstlisting}
\item ログアウト後,再ログイン(ユーザ名:user, パスワード live)すれば
\begin{lstlisting}
 ls Y
\end{lstlisting}
のようにアクセスできます.もちろんファイルマネージャPCManFMでも利用可能です.
\end{enumerate}

%\begin{thebibliography}{9}
% \bibitem{diskutil} Disk Utility 12.x: ディスク、CD、または DVD を複製する, \url{http://support.apple.com/kb/PH5846?viewlocale=ja_JP}
%\end{thebibliography}
\end{document}
